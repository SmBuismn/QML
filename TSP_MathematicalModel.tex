\documentclass[11pt,a4paper]{article}
\usepackage{amsmath}
\usepackage{amssymb}
\usepackage{anysize}
\usepackage{booktabs}
\usepackage{graphicx}
\usepackage{booktabs}
\usepackage{multirow}
\marginsize{1.5cm}{2cm}{2cm}{2cm}
\begin{document}
\title{\Large{Traveling Salesperson Problem - TSP - Mathematical Model }}
\date{}
\maketitle
\vspace{-0.5cm}
The notation used for this mathematical formulation is provided in Table \ref{tab:notation}. \vspace{-0.5cm} 
\begin{table}[h]
\caption{Notation} 
\centering
\label{tab:notation}
\begin{tabular}{lll}
\toprule
\multicolumn{2}{l}{\textbf{Sets and indices}} \\
\midrule							
$N$     & Set of nodes $i,j \in N$, $	\lvert N \rvert = n$, $0$ is the start/end node  \\
\midrule
\multicolumn{2}{l}{\textbf{Parameters}}   \\
\midrule
$c_{ij}$     & cost (or distance or time) from node $i$ to node $j$ \\
\midrule
\multicolumn{2}{l}{\textbf{Variables}}    \\ 
\midrule
$x_{ij}$    & binary variable, 1 if  route ($i$,$j$) is in the solution and 0 otherwise  \\
$u_i$  & order of node $i$ in the tour (subtour elimination) \\
\bottomrule 
\end{tabular}
%\noindent\rule{\textwidth}{0.4pt}
\end{table}

The mathematical formulation then follows as: 
%\vspace{-0.25cm} 
\begin{align}
\min \quad & \sum_{i \in N} \sum_{j \in N} c_{ij} x_{ij}\label{obj} &\\
\notag
\text{Subject to:} & &\\
& \sum_{j \in N} x_{ij} = 1 &\forall i \in N \label{outgoing} & \\
& \sum_{i \in N} x_{ij} = 1 &\forall j \in N \label{incoming} &\\
& u_1 = 1 \label{start_order} & &\\
& 2 \leq u_i \leq n & \forall i \in N, i>1 \label{order_range}& \\
& u_j - u_i + n x_{ij} \leq n - 1 & \forall i,j \in N, i \neq j \label{subtour_elimination} &\\
& x_{ij} \in \{0,1\} & \forall i \in N, j \in N \label{x_var} &\\
& u_i \geq 0 & \forall i \in N \label{u_var} &
\end{align}

Objective function (\ref{obj}) is written as a minimization of total cost (or distance or time). Constraints (\ref{outgoing}) ensure that each node has an outgoing arc and similarly constraints (\ref{incoming}) ensure that there is an incoming arc to ech node. Therefore, (\ref{outgoing}) and (\ref{incoming}) together maintain that each node is visited once and only once.   
Constraints (\ref{start_order})-(\ref{subtour_elimination}) are the subtour elimination constraints based on the MTZ formulation. Namely, constraint (\ref{start_order}) labels the starting node with an order of 1, constraints (\ref{order_range}) provide the range for the other nodes between 2 and the total number of nodes. Constraints (\ref{subtour_elimination}) then determine to order in relation to the $x$ variables by respecting the order of visit. Constraints (\ref{x_var})-(\ref{u_var}) define the variables as binary and continuous, respectively. 

%\begin{figure}[h]
%{\includegraphics[width=0.5\textwidth]{feas_reg.png}}
%\end{figure}

       
\end{document}