\documentclass[11pt,a4paper]{article}
\usepackage{amsmath}
\usepackage{amssymb}
\usepackage{anysize}
\usepackage{booktabs}
\usepackage{graphicx}
\usepackage{booktabs}
\usepackage{multirow}
\marginsize{1.5cm}{2cm}{2cm}{2cm}
\begin{document}
\title{\Large{Cargo Plane Example - Mathematical Model }}
\date{}
\maketitle
\vspace{-0.5cm}
The notation used for this mathematical formulation is provided in Table \ref{tab:notation}. \vspace{-0.5cm} 
\begin{table}[h]
\caption{Notation} 
\centering
\label{tab:notation}
\begin{tabular}{lll}
\toprule
\multicolumn{3}{l}{\textbf{Sets and indices}} \\
\midrule							
$I$     & Set for the cargo types & $i \in I$  \\
$J$     & Set for the compartments & $j \in J$ \\
$K$     & Set for the days & $k \in K$ \\
\midrule
\multicolumn{3}{l}{\textbf{Parameters}}   \\
\midrule
$m_{ik}$     & maximum quantity for cargo type $i$ on day $k$ & [ton] \\
$v_i$        & volume per ton of cargo type $i$ & [m$^3$/ton] \\
$p_i$        & profit per ton of cargo type $i$ & [euro/ton] \\
$w_j$        & weight capacity of compartment $j$ & [ton] \\
$s_j$        & volume capacity of compartment $j$ & [m$^3$] \\
\midrule
\multicolumn{3}{l}{\textbf{Variables}}    \\ 
\midrule
$x_{ijk}$    & load of cargo type $i$ in compartment $j$ on day $k$ & [ton]  \\
\bottomrule 
\end{tabular}
%\noindent\rule{\textwidth}{0.4pt}
\end{table}

The mathematical formulation then follows as: 
%\vspace{-0.25cm} 
\begin{align}
\max \quad & \sum_{i \in I} \sum_{j \in J} \sum_{k \in K} p_i x_{ijk}\label{obj} &\\
\notag
\text{Subject to:} & &\\
& \sum_{i \in I} v_i x_{ijk} \leq s_j &\forall j \in J, k \in K \label{volume_cons} \\
& \sum_{i \in I} x_{ijk} \leq w_j &\forall j \in J, k \in K \label{weight_cons} \\
& \sum_{d=1}^k \sum_{j \in J} x_{ijd} \leq \sum_{d=1}^k m_{id} &\forall i \in I, k \in K \label{cargo_cons} \\
& w_j \sum_{i \in I} x_{i1k} - w_1 \sum_{i \in I} x_{ijk} = 0 &\forall j \in J \setminus{\{1\}} , k \in K \label{balance_cons} \\
& x_{ijk} \geq 0 & \forall i \in I, j \in J, k \in K 
\end{align}

Objective function (\ref{obj}) is written as a maximization of total profit. Constraints (\ref{volume_cons}) and (\ref{weight_cons}) ensure that the volume and weight capacity of the compartments are respected for each day, respectively. Constraints (\ref{cargo_cons}) are for the cargo that becomes available for each type every day; namely you can not load more cargo than available on a given day. Weight balance of the airplane across compartments is maintained by constraints (\ref{balance_cons}) such that the weight in the compartments is proportional to their capacities. 

%\begin{figure}[h]
%{\includegraphics[width=0.5\textwidth]{feas_reg.png}}
%\end{figure}

       
\end{document}